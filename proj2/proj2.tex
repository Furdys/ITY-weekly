\documentclass[11pt,a4paper,twocolumn]{article}
\usepackage[left=1.5cm,text={18cm, 25cm},top=2.5cm]{geometry}
\usepackage{times}
\usepackage[utf8x]{inputenc}
\usepackage[IL2]{fontenc}
\usepackage{enumitem}
\usepackage{nameref}
\usepackage{amsfonts}
\usepackage{amsmath}
\usepackage{amsthm}

\newtheorem{thm}{Theorem}[section]
\newtheorem{mydef}[thm]{Definice}
\newtheorem{myalg}[thm]{Algoritmus}
\newtheorem{mysnt}{Věta}

\setlength{\parskip}{1pt}

\begin{document}

\begin{titlepage}
	\begin{center}
		\linespread{0}
		\Huge
			\textsc{Fakulta informačních technologií \\
			Vysoké učení technické v~Brně}
			\\[84mm]
		\LARGE
			Typografie a~publikování -- 2. projekt \\
			Sazba dokumentů a~matematických výrazů
			\vfill
	\end{center}
	{\Large 2017 \hfill Jiří Furda}
\end{titlepage}

\label{page}

\section*{Úvod}
V~této úloze si vyzkoušíme sazbu titulní strany, matematických vzorců, prostředí a~dalších textových struktur obvyklých pro technicky zaměřené texty (například rovnice \eqref{eq} nebo definice \ref{mydef} na straně \pageref{page}).

Na titulní straně je využito sázení nadpisu podle op\-tic\-ké\-ho středu s~využitím zlatého řezu. Tento postup byl probírán na přednášce.


\section{Matematický text}

Nejprve se podíváme na sázení matematických symbolů a~výrazů v~plynulém textu. Pro množinu $V$ označuje $\mathrm{card}(V)$ kardinalitu $V$.
Pro množinu $V$ reprezentuje $V^*$ volný monoid generovaný množinou $V$ s~operací kon\-ka\-te\-na\-ce.
Prvek identity ve volném monoidu $V^*$ značíme symbolem $\varepsilon$ .
Nechť $V^+ = V^* - \{ \varepsilon \}$ Algebraicky je tedy $V^+$ volná pologrupa generovaná množinou $V$ s~operací konkatenace.
Konečnou neprázdnou množinu $V$ nazvěme \emph{abeceda}.
Pro $w \in V^* $ označuje $|w|$ délku řetězce $w$. Pro $W \subseteq V$ označuje $\mathrm{occur}(w,W)$ počet výskytů symbolů z~$W$ v~řetězci $w$ a~$\mathrm{sym}(w,i)$ určuje $i$-tý symbol řetězce $w$; například $\mathrm{sym}(abcd,3) = c$.

Nyní zkusíme sazbu definic a~vět s~využitím balíku \texttt{amsthm}.

\begin{mydef}\label{mydef}
Bezkontextová gramatika \normalfont je čtveřice $G = (V,T,P,S)$, kde $V$ je totální abeceda, $T \subseteq V$ je a\-be\-ce\-da ter\-mi\-ná\-lů, $S \in (V - T)$ je startující symbol~a $P$ je konečná množina \emph{pravidel} tvaru $q: A \rightarrow \alpha$, kde $A \in (V - T)$, $\alpha \in V^*$ a~$q$ je návěští tohoto pravidla. Nechť $N = V - T$ značí abecedu neterminálů.
Pokud $q: A \rightarrow \alpha \in P$, $\gamma$, $\delta$ provádí derivační krok z $\gamma A \delta$ do $\gamma \alpha \delta$ podle pravidla $q: A \rightarrow \alpha$, symbolicky píšeme 
$\gamma A \delta \Rightarrow \gamma \alpha \delta [q: A \rightarrow \alpha]$ nebo zjednodušeně $\gamma A \delta \Rightarrow \gamma \alpha \delta$. Standardním způsobem definujeme $\Rightarrow^m$, kde $m \geq 0$. Dále definujeme tranzitivní uzávěr $\Rightarrow^+$ a tranzitivně-reflexivní uzávěr $\Rightarrow^*$.
\end{mydef}

Algoritmus můžeme uvádět podobně jako definice textově, nebo využít pseudokódu vysázeného ve vhodném prostředí (například \texttt{algorithm2e}).

\begin{myalg}
	Algoritmus pro ověření bezkontextovosti gramatiky. Mějme gramatiku $G = (N, T, P, S)$.
	\begin{enumerate}
		\item
			Pro každé pravidlo $p \in P$ proveď test, zda $p$ na levé straně obsahuje právě jeden symbol z $N$. \label{mystep}
		\item
			Pokud všechna pravidla splňují podmínku z kroku \ref{mystep}, tak je gramatika $G$ bezkontextová.
	\end{enumerate}   
\end{myalg}

\begin{mydef}
	Jazyk \normalfont definovaný gramatikou $G$ definujeme jako $\mathrm{L}(G) = \{w \in T^* | S \Rightarrow^* w \}$.
\end{mydef}

\subsection{Podsekce obsahující větu}

\begin{mydef}
	\normalfont
		Nechť $L$ je libovolný jazyk. $L$ je \emph{bezkontextový jazyk}, když a~jen když $L = \mathrm{L}(G)$, kde $G$ je libovolná bezkontextová gramatika.
\end{mydef}

\begin{mydef}
	\normalfont
		Množinu $\mathcal{L}_{CF} = \{L|L $ je bezkontektový jazyk\} nazýváme \emph{třídou bezkontextových jazyků}.
\end{mydef}

\begin{mysnt}\label{snt}
	Nechť $L_{abc} = \{ a^n b^n c^n | n \geq 0 \}$ Platí, že $L_{abc} \not\in \mathcal{L}_{CF}$
\end{mysnt}

\begin{proof}[Důkaz]
	Důkaz se provede pomocí Pumping lemma pro bezkontextové jazyky, kdy ukážeme, že není možné, aby platilo, což bude implikovat pravdivost věty \ref{snt}.
\end{proof}

\section{Rovnice a~odkazy}

Složitější matematické formulace sázíme mimo plynulý text. Lze umístit několik výrazů na jeden řádek, ale pak je třeba tyto vhodně oddělit, například příkazem \verb|\quad|. 


\begin{equation*}
	\sqrt[x^3]{y^3_0} \quad
	\mathbb{N} = \{0,1,2,...\} \quad
	x^{y^y} \ne x^{yy} \quad
	x^{y^y} \not\equiv x^{yy}
\end{equation*}

V~rovnici (...) jsou využity tři typy závorek s~různou explicitně definovanou velikostí.

\begin{eqnarray}
	\bigg\{ \Big[ \big(a + b * c \big) \Big] ^d + 1 \bigg\}  & = & x \label{eq}\\
	\lim_{x\rightarrow\infty} \frac{ \sin^2x + \cos^2x}{4}  & = & y \nonumber
\end{eqnarray}

V~této větě vidíme, jak vypadá implicitní vysázení li\-mi\-ty $\lim_{n \rightarrow \infty} f(n)$ v~normálním odstavci textu. Podobně je to i~s~dalšími symboly jako $\sum^n_1$ či $\bigcup_{A \in B}$ . V~případě vzorce $\lim_{x \rightarrow 0} \limits \frac{\sin x}{x} = 1$  jsme si vynutili méně úspornou sazbu příkazem \verb|\limits|.

\begin{eqnarray}
	\int\limits_a^b f(x) \mathrm{d}x & = & - \int_b^a f(x) \mathrm{d}x \\
	\Big( \sqrt[5]{x^3} \Big) ' = \Big( x^{\frac{4}{5}} \Big) ' & = &
	\frac{4}{5} x^{-\frac{1}{5}} = \frac{4}{5 \sqrt{5}{x} } \\
	\overline{\overline{A \lor B}} & = & \overline{\overline{A} \land \overline{B}}
\end{eqnarray}
\section{Matice}

Pro sázení matic se velmi často používá prostředí \texttt{array} a~závorky (\verb|\left|, \verb|\right|). 
$$
\left(
 \begin{array}{cc}
  a + b & b - a \\
  \widehat{\xi + \omega} & \hat{\pi} \\
  \vec{a} & \overleftrightarrow{AC} \\
  0 & \beta  
 \end{array}
\right)
$$ 

$$
A =
\left\|
	\begin{array}{cccc}
		a_{11} & a_{12} & \cdots & a_{1n} \\
		a_{21} & a_{22} & \cdots & a_{2n} \\
		\vdots  & \vdots  & \ddots & \vdots  \\
		a_{m1} & a_{m2} & \cdots & a_{mn} 
	\end{array}
\right\|
$$

$$
\left|
	\begin{array}{cc}
		t & u \\
		v & w
	\end{array}
\right|
= tw-uv
$$

Prostředí array lze úspěšně využít i jinde.

$$
\binom{n}{k} =
\left\{
	\begin{array}{c l}
		\frac{n!}{k!(n-k)!} & \text{pro } 0 \leq k \leq n \\
		0 & \text{pro } k < 0 \text{ nebo } k > n
	\end{array}
\right. $$

\section{Závěrem}

V~případě, že budete potřebovat vyjádřit matematickou konstrukci nebo symbol a~nebude se Vám dařit jej nalézt v~samotném \LaTeX u, doporučuji prostudovat možnosti balíku maker \AmS-\LaTeX.
Analogická poučka platí o\-bec\-ně pro ja\-kou\-koli konstrukci v~\TeX u.

\end{document}
