\documentclass[11pt,a4paper,twocolumn]{article}
\usepackage[left=2cm,text={17cm, 24cm},top=2.5cm]{geometry}
\usepackage[utf8x]{inputenc}
\usepackage[czech]{babel}
\usepackage{times}
\usepackage{changepage}
\usepackage[IL2]{fontenc}

\title{Typografie a publikování \\ 1. projekt}
\author{Jiří Furda \\ xfurda00@stud.fit.vutbr.cz}
\date{}

\begin{document}

\maketitle

\section{Hladká sazba}

Hladká sazba je sazba z~jednoho stupně, druhu a~řezu pí­sma sázená na stanovenou šířku odstavce. Skládá se z~odstavců, které obvykle začínají­ zarážkou, ale mohou být sázeny i~bez zarážky -- rozhodují­cí­ je celková grafická úprava. Odstavce jsou ukončeny východovou řádkou. Věty nesmějí začínat číslicí.

Barevné zvýraznění­, podtrhávání­ slov či různé velikosti písma vybraných slov se zde také nepoužívá. Hladká sazba je určena především pro delší­ texty, jako je napří­klad beletrie. Porušení­ konzistence sazby působí v~textu rušivě a~unavuje čtenářův zrak.

\section{Smíšená sazba} 

Smíšená sazba má o~něco volnější­ pravidla než hladká sazba. Nejčastěji se klasická hladká sazba doplňuje o~další řezy pí­sma pro zvýraznění­ důležitých pojmů. Existuje \uv{pravidlo}:

\vspace{3mm}
\begin{adjustwidth}{8mm}{8mm}
\hspace{6mm}Čí­m ví­ce \textbf{druhů, \emph{řezů},}  {\scriptsize velikostí}, barev pí­sma a~jiných efektů použijeme, tí­m \emph{profesionálněji} bude  dokument vypadat. Čtenář tím bude vždy {\Huge nadšen!}
\end{adjustwidth}
\vspace{3mm}

\textsc{Tí­mto pravidlem se \underline{nikdy} nesmí­te ří­dit.} Příliš časté zvýrazňování textových elementů a~změny velikosti {\tiny pí­sma} jsou {\LARGE známkou} \textbf{\huge a\-ma\-tér\-is\-mu} autora a~působí­ \textbf{velmi} \emph{rušivě}. Dobře na\-vr\-že\-ný dokument nemá obsahovat ví­ce než 4 řezy či druhy pí­sma. \texttt{Dobře navržený dokument je decentní­, ne chaotický}.

Důležitým znakem správně vysázeného dokumentu je konzistentní použí­vání­ různých druhů zvýraznění­. To napří­klad může znamenat, že \textbf{tučný řez} pí­sma bude vyhrazen pouze pro klíčová slova, \emph{skloněný řez} pouze pro doposud neznámé pojmy a~nebude se to míchat. Skloněný řez nepůsobí­ tak rušivě a~použí­vá se častěji. V~{\LaTeX}u jej sází­me raději pří­kazem \verb|\emph{text}| než \verb|\textit{text}|.

Smíšená sazba se nejčastěji používá pro sazbu vě\-dec\-kých článků a~technických zpráv. U delší­ch do\-ku\-ment\-ů vědeckého či technického charakteru je zvykem upozornit čtenáře na význam různých typů zvý\-raz\-ně\-ní­ v~úvodní­ kapitole.

\section{České odlišnosti}

Česká sazba se oproti okolní­mu světu v~některých as\-pekt\-ech mí­rně liší­. Jednou z~odlišností je sazba uvozovek. Uvozovky se v~češtině použí­vají­ převážně pro zobrazení­ pří­mé řeči. V~menší­ míře se použí­vají­ také pro zvý\-raz\-ně\-ní­ přezdí­vek a~ironie. V~češtině se použí­vá tento \uv{\textbf{typ uvozovek}} namí­sto anglických ‘‘uvozovek’’. Lze je sázet připravenými příkazy nebo při použití UTF-8 kódování i~přímo.

Ve smíšené sazbě se řez uvozovek ří­dí­ řezem prv\-ní\-­ho uvozovaného slova. Pokud je uvozována ce\-lá věta, sází­ se koncová tečka před u\-vo\-zov\-ku, pokud se uvozuje slovo nebo část věty, sází­ se tečka za u\-voz\-ov\-ku.

Druhou odlišností je pravidlo pro sázení­ konců řád\-ků. V~české sazbě by řádek neměl končit osamocenou jednopí­smennou předložkou nebo spojkou. Spojkou \uv{a} končit může při sazbě do 25 liter. Abychom {\LaTeX}u zabránili v~sázení­ osamocených předložek, vkládáme mezi předložku a~slovo \textbf{nezlomitelnou mezeru} znakem \~{} (vlnka, tilda). Pro automatické do\-pl\-ně\-ní vlnek slouží­ volně šiřitelný program \emph{vlna} od pana Olšáka\footnote{Viz http://petr.olsak.net/ftp/olsak/vlna/.}.

\end{document}
